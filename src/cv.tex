\documentclass[10pt,a4paper]{moderncv}
\moderncvtheme[blue]{classic}  
\usepackage[top=0.8cm, bottom=0.8cm, left=0.8cm, right=0.8cm]{geometry}
\usepackage[frenchb]{babel}
\usepackage[utf8]{inputenc}

% Largeur de la colonne pour les dates
\setlength{\hintscolumnwidth}{4.0cm}

\firstname{Bastien}
\familyname{Levesque}     
\title{Ingénieur INSA - Architecte des Systèmes d'Information}
\address{9 Rue des Pinsons}{76570 PAVILLY}    
\email{bastienlevesque01@gmail.com}                      
\mobile{06 59 21 52 53} 
\extrainfo{23 ans -- Permis B}
\photo[80pt][0pt]{images/bastien.png}
\begin{document}
\rmfamily
\maketitle

\section{Expériences professionnelles}

\cventry{Novembre 2017 à Aujourd'hui}{Ingénieur FullStack}{Event Bots}{Rouen}{France}{
\textit{Event Bots} produit les robots humanoïdes d'accueil Tiki et Maava. \begin{itemize}%
\item Recherche et développement : \begin{itemize}
	\item Navigation autonome, 
	\item Détection faciale,
	\item Reconnaissance vocale,\end{itemize}
\item Développement d'applications pour tablettes embarquées ;
\item Mise en place de procédures/bancs de test de cartes électroniques ;
\item Travail avec le client pour les spécifications des projets;
\item Gestion de projet.
\\\end{itemize}
Mots clés : \textbf{CoffeeScript, C, Blade, Stylus, WebPack, OpenCV, ROS, Python}.\\}

\cventry{Janvier 2017 à Juillet 2017}{Stage ingénieur}{Stanley Robotics}{Paris}{France}{
Amélioration de la simulation du robot de l'entreprise : \begin{itemize}%
\item Cahier des charges ;
\item États de l'art ;
\item Refonte architecturale de la simulation ;
\item Développement de nouvelles fonctionnalités ;
\item Tests.\\\end{itemize}
Mots clés : \textbf{ROS, Gazebo, V-Rep, C++, Python}.\\
% Contact - Aurélien Cord (CTO) : \textit{aurelien.cord@stanley-robotics.com}.\\}
}

\cventry{Janvier 2016 à Janvier 2017}{Chef de projet (9 personnes)}{pOwOw.box}{Projet Insa Certifié ISO 9001:2015 Rouen}{France}{
Développement d'un gestionnaire d'agents mobiles expressifs pour Android.\\
Mots clés : \textbf{C\#, Unity3D}.\\}

\cventry{Juin 2016 à Août 2016}{Stage de spécialité}{Atracsys}{Lausanne}{Suisse}{
Développement et auto-gestion de 3 projets : \begin{itemize}%
\item CMS Web : \textbf{HTML, CSS, JS, PHP} ;
\item Jeux vidéo pour tables multi-touch : \textbf{C\#m : Matlab}.\\\end{itemize}
% Contact - Gaëtan Marti (CEO) : \textit{gaetan.marti@atracsys.com}.\\}
}

% \cventry{Avril 2014 à Juin 2014}{Stage technicien}{Réseau et Transport d'Électricité}{La Vaupalière}{France}{}

% \cventry{Juin 2013 à Juillet 2013}{Travail saisonnier}{SNCF}{Rouen}{France}{}

\section{Formations \& Diplômes}

\cventry{2014 -- 2017}{Diplôme d'ingénieur, Architecture des Systèmes d'Information}{Institut National des Sciences Appliquées}{Rouen}{France}{\begin{itemize}%
\item Programmation orientée objet ;
\item Conception et développement de logiciels ;
\item Algorithmique ;
\item Vision par ordinateur ;
\item Réseaux informatiques ;
\item Gestion de projet (Agile et ISO 9001:2015).
\end{itemize}}

\cventry{2012 -- 2014}{DUT, Génie Électrique et Informatique Industrielle}{Université}{Rouen}{France}{}

% \cventry{2009 -- 2012}{Baccaulauréat Scientifique mention Bien}{Lycée Thomas Corneille}{Barentin}{France}{}


\section{Compétences techniques}
\cvitem{Développement}{C\#, C++, Python, JAVA, Bash, Technologies Web.}
\cvitem{Logiciels}{Git, Adobe Photoshop/Première/After Effect, Unity3D, ROS, Gazebo, V-Rep.}
% \cvitem{Logiciels}{Git, ROS, Gazebo, V-Rep, Unity3D, Matlab, SolidWorks, Adobe Photoshop/Première/After Effect.}
\section{Compétences linguistiques}
\cvlanguage{Anglais}{Toeic 820 / 990.}{}
\cvlanguage{Espagnol}{Niveau B2.}{}

\section{Centres d'intérêt}
\cvitem{Musique}{Écoute, interprétation, création et production. Organisation de concerts. Pratique de multiples instruments.}
\cvitem{Sports}{Skateboard, randonnée, natation.}
\cvitem{Divers}{Création audiovisuelle, e-sport, photographie, réalité virtuelle, robotique et exploration spatiale.}

\end{document}
